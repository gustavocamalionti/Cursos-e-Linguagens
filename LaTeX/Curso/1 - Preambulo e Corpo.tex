\documentclass[a4paper, 12pt]{article}
\usepackage[top=2cm,bottom=2cm,left=3cm,right=3cm]{geometry}
\usepackage[utf8]{inputenc}

\begin{document}
\begin{center}
\textbf{Equa\c c\~ao polinomial do 2$^\circ$ grau.}
\end{center}
\begin{flushright}
\textit{Equação polinomial do 2$^\circ$ grau.}
\end{flushright}

\begin{flushleft}
\underline{Equação polinomial do 2$^\circ$ grau}
\end{flushleft}

\begin{center}
\textit{\textbf{\underline{Equação polinomial do 2$^\circ$}}}
\end{center}

Uma equação da forma $$ax^2 + bx + c = 0,$$ $a \neq 0$ será chamada de equação polinomial do 2$^\circ$ grau.
$$x=\frac{ -b \pm \sqrt{b^2 - 4.a.c}}{2a} $$
\end{document}